\documentclass{tufte-handout}

\usepackage[cxx]{211lab}

\LabInfo{5}{Welcome to C++}

\begin{document}

\maketitle

Today we begin programming in C++ and the GE211 game engine in a minimal
example game. The game is quite simple: The player controls two circles
on the screen---one with the mouse and one with the keyboard---and when
the two circles overlap, one changes color. As you will see, however, it
comes with a bug.

Before you can get started, you’ll need to install a C++ and GE211
development environment. This means you’ll be setting up
%\marginpar{The C++ setup instructions are \href{\CxxSetupUrl}{here}.}
a C++ compiler,
the CLion IDE, and the SDL2 graphics libraries.

\CxxPrelims*

\section{The game}

\subsection{Stating a program}

To select your build target, use the dropdown menu in the top right of the
CLion IDE. You should select ``circle_game''. To compile and run code, click
the green ``play'' button in the right of toolbar. After compiling, the game
window should automatically pop up.

\subsection{Broken control}

Currently, there is an bug in this code. Run the program by clicking the
green “play” button in the toolbar. Then try to control the circle with
your left and right arrow keys, and the big circle should likely move
in the opposite direction of what you intend. A bug!.

Open the project viewer by clicking ``Project'' on the right side of the CLion
IDE and locate the code for this---hint: look in the ^src/model.cxx^---and fix
it. Run the code again to verify your fix.

There are also test cases for checking the model's movement, so when you are
done try running your code against the tests. To build the tests, choose
``model_test'' from the dropdown menu and again click the green ``play''
button. This time a window won't pop up, but rather the test results will
appear at the bottom of the CLion IDE.

\subsection{Up and down}

As you have seen, the circle that is controlled by the keyboard only
moves horizontally right now. There are two \emph{member functions} for
moving the large circle up and down,
^Model::move_^\-^large_^\-^circle_^\-^up()^ and
^Model::move_^\-^large_^\-^circle_^\-^down()^, declared as part of the
^Model^ struct in \filename{src/model.hxx}. Write their definitions in
\filename{src/model.cxx}, following the pattern of the similar member functions.
Then connect them to the keyboard by modifying the ^Game::on_key(Key)^
member function in \filename{ui.cxx} to handle additional keys.

Be sure to run the ``circle_game'' code after you modify it and make sure
that it works. You might get the directions wrong on the up and down keys
the first time you try!

\subsection{Click, not hover}

Currently, the position of the smaller circle tracks the position of the
mouse. However, what if we want the game to only update the position of
smaller circle when we click? To detect mouse clicks, you will have to
override the
\href{https://tov.github.io/ge211/classge211_1_1_abstract__game.html#ae010e7d2f2f51229f45eb2046e1a5bde}
{^Abstract_game::on_mouse_down(Mouse_button, Position)^}
function in the ^Game^ struct.
\marginpar{The function signature
is a link to the function’s documentation.}

You'll need to modify the code in \filename{ui.cxx} and \filename{ui.hxx} to
remove the ^on_mouse_move()^ function and instead override ^on_mouse_down^ with
the proper arguments. After you do, run the game again and verify successful
behavior.

\subsection{Testing}

The current tests include a few examples that should pass
for your code. Fill in the two tests for moving the large circle up and down
to verify it works as expected.

There is a test case that checks that ^Model::overlapped()^ will return ^true^
when the two circles are touching. Fill in the final test for checking when the
circles aren't touching.

\section{Other things to try}

Documentation for the \href{https://tov.github.io/ge211/classge211_1_1_abstract__game.html}{Abstract Game Class}.

\begin{itemize}

  \item Make the small circle change colors when it's touching
    the large circle.

  \item Make the small circle change to a different color when it's
    touching the edge of the window.

  \item Make a circle change size when touching the other.

  \item Let the user change the colors by pressing different keys on the
    keyboard.

  \item Change a circle to a rectangle.

  % \item Change a circle to an image you find on the internet.

\end{itemize}

\end{document}
